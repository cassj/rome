\subsubsection{Overview}
\label{sec:model_overview}

\paragraph{}
The ROME model is implemented as a MySQL database with DBIx::Class object-relational mapping. The main base DBIx::Class schema class is ROMEDB.pm, which is decoupled from Catalyst and can be used by other applications to access the database.  Each table in the database has a corresponding class in \texttt{lib/ROMEDB/} which defines how to generate a perl DBIx::Class object from elements of that table. The ROMEDB schema is made available to Catalyst via ROME::Model::ROMEDB, a subclass of Catalyst::Model::DBIC::Schema. The configuration options: which schemas to use, connection details and so on, are picked up from the \texttt{rome.yml} config file. The schema can therefore be accessed via the Catalyst context object within controller actions and view templates. 

\paragraph{}
The easiest way to get an overview of the model structure is to consider the main entities involved and the conceptual relationships between them, as illustrated in figure \ref{fig:modover}. In brief, there are \texttt{Users}, who have \texttt{Experiments}, which are composed of \texttt{Jobs}. \texttt{Jobs} accept \texttt{Datafiles} as input and generate \texttt{Datafiles} as output. Entries in the \texttt{Process} table describe \texttt{Processes} which can be run as \texttt{Jobs}. A \texttt{Process} has associated \texttt{Parameters}. A \texttt{Process} also defines the \texttt{Datatype} of the \texttt{Datafiles} it can \texttt{Accept} and \texttt{Create}. A \texttt{Job} can be considered a specific instance of a \texttt{Process}, so where a \texttt{Process} has a description of the \texttt{Datatypes} it accepts and creates, a \texttt{Job} has actual \texttt{Input Datafiles} and \texttt{Output Datafiles} of the appropriate \texttt{Datatype}. Where a \texttt{Process} has associated \texttt{Parameters} a \texttt{Job} has \texttt{Arguments}, the actual \texttt{Parameter} values used for the \texttt{Job}. A \texttt{Process} also defines a \texttt{Processor} which is capable of running \texttt{Jobs} generated from that \texttt{Process}. Currently the only available \texttt{Processor} is \texttt{ROME::Processor::R}. \texttt{Jobs} are entered in the \texttt{Queue} to be run. A seperate job scheduler is responsible for checking the contents of the queue and running the jobs. \texttt{Processes} belong to \texttt{Components}. A \texttt{Component} can be thought of as a ROME add-on package and comprises model,view and controller elements. 

% \paragraph{}
% \texttt{Workflows} can be defined as a series of \texttt{Processes} which can be chained together by the \texttt{Datafiles} they accept and create. As a \texttt{Workflow} is a chained set of \texttt{Processes} and their associated \texttt{Parameters}, a \texttt{Career} is a chained set of \texttt{Jobs} and their associated \texttt{Arguments}.

\paragraph{}
Each of the database tables has a corresponding DBIx::Class class through which the model can be accessed in ROME through which information in the database can be created, retrieved, updated and deleted. Relationships between tables are also modelled in the DBIx::Class classes which allows related objects to be retrieved by simply calling a method. For example all of the jobs belonging to a particular experiment can be retrieved as a result-set from an experiment object as shown below. In scalar context, this will return a resultset that can be used as an iterator. In array context an array of datafile objects will be returned. ROME model classes contain perldoc describing their use and available methods. Further general information about using DBIx::Class can be found in the manual at \url{http://search.cpan.org/~ash/DBIx-Class-0.08010/lib/DBIx/Class/Manual.pod}

\begin{scriptsize}
\begin{verbatim}
my $jobs = $experiment->jobs;
while (my $this_job = $jobs->next){
  #do something with the jobs
}
\end{verbatim}
\end{scriptsize}



\paragraph{}
Figure \ref{fig:modover} provides an overview of the entire model and the SQL definition can be found in appendix \ref{ch:rome_schema}. The following sections describe the model in more detail.


\begin{figure}
\centering
\caption{ROME Overview}\label{fig:modover}
\includegraphics[scale=0.6]{images/core_overview}
\end{figure}

\begin{figure}
\centering
\caption{ROME Data Model}\label{fig:modover}
\includegraphics[scale=0.32, angle=90]{images/modover}
\end{figure}

\clearpage
