\section{Workgroups}
\label{workgroups}

\subsection{The database table: workgroup}

\begin{verbatim}
create table workgroup(
  name VARCHAR(100) NOT NULL PRIMARY KEY,
  description TEXT,
  leader CHAR(50) NOT NULL REFERENCES person(username)
);
\end{verbatim}

\paragraph{}
Workgroups allow users to share their experiments with a limited number of other users. Any user can create a workgroup via the interface by going to My Account > My Workgroups. They must define a (unique) name for their workgroup and optionally a brief description. They will automatically become the leader of that workgroup. Once the name of a group has been set, it cannot be changed. The description can be altered at any time by the group leader (again from the My Workgroups page). The group leader may also hand over leadership to another member of the group if they wish, meaning that the new leader takes over their update rights.

\subsubsection{Inviting members and joing workgroups}

\paragraph{}
The leader of a workgroup may invite any other users to join that group (from the My Workgroups page). Invited users will receive email notification and will be able to accept or decline the invitation from their own Workgroups page. Users may also request membership of a workgroup from their My Workgroups page, in which case the group leader will recieve email notification and must permit or deny the request.
