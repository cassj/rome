\subsubsection{Skins}
\label{sec:view_skins}

\paragraph{}
Although the way in which data is presented to the user is, to some degree, component-specific, the overall look and feel of ROME is easily customisable. Simple changes, such as modifying the colour-scheme, can be achieved trivially in the config files. More dramatic changes can be achieved by designing new skins. Figure \ref{fig:default_skin} shows the default ROME skin (based on the default Catalyst style). Figure \ref{fig:lpc_skin} shows the skin used for the London Pain Consortium. Specifying a particular skin is as simple as altering \texttt{skin: 'default'} to the name of the new skin in \texttt{rome.yml}.  The view module (\texttt{ROME/View/TT.pm}) is set up to search the directory specified by the config file for templates.


\clearpage
\begin{figure}
\centering
\caption{Default ROME skin}\label{fig:default_skin}
\includegraphics[scale=0.2]{images/default_skin}
\centering
\caption{LPC ROME Skin}\label{fig:lpc_skin}
\includegraphics[scale=0.2]{images/lpc_skin}
\end{figure}

\paragraph{}
Multiple skins may be stored in \texttt{root/skins} and the selected skin is specified in the main config file.

%  A skin consists of two directories, \texttt{config} and \texttt{site}.

% \paragraph{}
% The config files are Template Toolkit (section \ref{sec:tt}) files which define configuration settings for the templates in \texttt{site}. The default skin has two config files, \texttt{main}, and \texttt{col}. The various settings these files define can be accessed in the site templates as \texttt{site.setting\_name}, for example \texttt{site.col.page}.  Altering the colour scheme for ROME, can be acheived simply by altering the colours in \texttt{col}. \texttt{main} is used to define any other global template code, though this gets little use in the default skin. 
% 
